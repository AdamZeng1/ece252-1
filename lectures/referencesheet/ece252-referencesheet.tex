\documentclass[legalpaper,10pt]{article}

\usepackage{titling}
\usepackage{listings}
\usepackage{url}
\usepackage{setspace}
\usepackage{subfig}
\usepackage{sectsty}
\usepackage{pdfpages}
\usepackage{colortbl}
\usepackage{multirow}
\usepackage{multicol}
\usepackage{relsize}
\usepackage{amsmath}
\usepackage{ifsym}
\usepackage{ulem}
\usepackage{fancyvrb}
\usepackage{amsmath,amssymb,amsthm,graphicx,xspace}
\usepackage[titlenotnumbered,noend,noline]{algorithm2e}
\usepackage[compact]{titlesec}
\usepackage{XCharter}
\usepackage[T1]{fontenc}
\usepackage[scaled]{beramono}
\usepackage{tikz}
\usetikzlibrary{arrows,automata,shapes,trees,matrix,chains,scopes,positioning,calc,patterns}
\tikzstyle{block} = [rectangle, draw, fill=blue!20, 
    text width=2.5em, text centered, rounded corners, minimum height=2em]
\tikzstyle{bw} = [rectangle, draw, fill=blue!20, 
    text width=4em, text centered, rounded corners, minimum height=2em]

\definecolor{namerow}{cmyk}{.40,.40,.40,.40}
\definecolor{namecol}{cmyk}{.40,.40,.40,.40}

\let\LaTeXtitle\title
\renewcommand{\title}[1]{\LaTeXtitle{\textsf{#1}}}

\lstset{basicstyle=\footnotesize\ttfamily,breaklines=true}

\newcommand{\handout}[5]{
  \noindent
  \begin{center}
  \framebox{
    \vbox{
      \hbox to 5.78in { {\bf ECE254: Operating Systems and Systems Programming } \hfill #2 }
      \vspace{4mm}
      \hbox to 5.78in { {\Large \hfill #4  \hfill} }
      \vspace{2mm}
      \hbox to 5.78in { {\em #3 \hfill} }
    }
  }
  \end{center}
  \vspace*{4mm}
}

\newcommand{\lecture}[3]{\handout{#1}{#2}{#3}{Lecture #1}}
\newcommand{\tuple}[1]{\ensuremath{\left\langle #1 \right\rangle}\xspace}

\addtolength{\oddsidemargin}{-1.000in}
\addtolength{\evensidemargin}{-0.500in}
\addtolength{\textwidth}{2.0in}
\addtolength{\topmargin}{-1.000in}
\addtolength{\textheight}{1.75in}
\addtolength{\parskip}{\baselineskip}
\setlength{\parindent}{0in}
\renewcommand{\baselinestretch}{1.5}
\newcommand{\term}{Spring 2015}

\singlespace

\newcommand{\vclass}{ECE 252 Spring 2019}
\newcommand{\vtitle}{Function \& Library Reference}
%\newcommand{\vpage}{1}
%\newcommand{\vqcount}{2 }
\newcommand{\rectangle}[3]{\tikz \draw (0,0) rectangle (#1,#2) node[pos=.5]{#3};}
\newcommand{\multiline}[2][c]{%
  \begin{tabular}[#1]{@{}c@{}}#2\end{tabular}}

\usepackage{tikz}
\usepackage{enumitem}
\usepackage{fancyhdr}
 
\pagestyle{fancy}
\usepackage{lastpage}
\lhead{Instructor: J. Zarnett}
\chead{}
\rhead{\bfseries \vtitle: \vclass }
%\lfoot{}
\setlength{\parindent}{0cm}
\setlength{\parskip}{0.3cm}

%\newcommand{\answer}[1]{\fbox{\parbox{.8\linewidth}{\textbf{Answer guide:} \\#1}}}
%\renewcommand{\answer}[1]{}

\begin{document}

\section*{Function \& Library Reference}

\lstset{numbers=none}

\begin{lstlisting}[language=C]
void* malloc( int num_bytes );
void* calloc( size_t num, size_t size );
void *realloc( void *ptr, size_t new_size );
void free( void* p );
int atoi( const char *s ); /* Convert string to int */
int printf( const char * fmt, ... ); /* %d int, %lu unsigned long, %f double,  %s string, \n newline, ...=args */
void* memcpy( void *destination, const void *source, size_t num_bytes ); /* Returns destination */
void *memset( void *mem, int value, size_t num_bytes ); /* Set memory to 0 */
size_t strlen( const char * string ); /* returns length of null-terminated string, not counting the terminator */
int strncmp ( const char * str1, const char * str2, size_t max_size ); /* return 0 if strings equal */
\end{lstlisting}

\subsubsection*{Files}

\begin{lstlisting}[language=C]
int open( const char *filename, int flags );  /* Returns a file descriptor if successful, -1 on error */
ssize_t read( int file_descriptor, void *buffer, size_t count ); /* Returns number of bytes read */
ssize_t write( int file_descriptor, const void *buffer, size_t count ); /* Returns number of bytes written */
int rename( const char *old_filename, const char *new_filename ); /* Returns 0 on success */
int close( int file_descriptor ); 
FILE* fopen( const char *filename, const char *mode );
int fclose( FILE* f );
size_t fread( void* buffer, size_t size, size_t numbytes, FILE* f );
size_t fwrite( const void *ptr, size_t size, size_t numbytes, FILE* f);
int fprintf( FILE* f, const char* format, ... ); /* Arguments go in the ...  */
int fscanf( FILE* f, const char* format, ... ); /* Arguments go in the ...  */
int fileno( FILE* f ); /* Convert FILE* to file descriptor */
int fseek ( FILE * f, long int offset, int origin ); /*SEEK_SET from file start. SEEK_CUR from current loc.*/
int remove( const char* filename );
int flock( int file_descriptor, int type ); /* LOCK_EX, LOCK_SH, or LOCK_UN for type */
int fcntl( int file_descriptor, int command, ... /* struct flock * flockptr */ );
int lockf( int file_descriptor, int command, off_t length );

struct flock {
  short l_type; /* F_RDLCK, F_WRLCK, or F_UNLCK */
  short l_whence; /* SEEK_SET, SEEK_CUR, or SEEK_END */
  off_t l_start; /* offset in bytes, relative to l_whence */ 
  off_t l_len; /* length, in bytes; 0 means lock to EOF */ 
  pid_t l_pid; /* returned with F_GETLK */
};

int inotify_init( ); /* Returns file descriptor referring to the struct */ 
int inotify_add_watch( int fd, const char* pathname, uint32_t mask );
int inotify_rm_watch( int fd, uint32_t wd );

struct inotify_event {
  int      wd;       /* Watch descriptor */
  uint32_t mask;     /* Mask describing event */
  uint32_t cookie;   /* Unique cookie associating related events (for rename(2)) */  
  uint32_t len;      /* Size of name field */
  char     name[];   /* Optional null-terminated name */
};

\end{lstlisting}

When opening a file with \texttt{fopen()}, the options are:\\
\begin{tabular}{l|l}
	\textbf{Mode} & \textbf{Meaning} \\ \hline
	\texttt{r} & Open the file read-only. \\ \hline
	\texttt{w} & Open the file for writing (create if needed) \\ \hline
	\texttt{a} & Open the file for appending (create if needed) \\ \hline
	\texttt{r+} & Open the file for reading, from the start \\ \hline
	\texttt{w+} & Open the file for writing (overwrite) \\ \hline
	\texttt{a+} & Open the file for reading and writing, append if exists \\
\end{tabular}


For \texttt{open()} the following flags may be used for the \texttt{flags} parameter, and can be combined with \texttt{|} (bitwise OR):\\
\begin{tabular}{l|l}
	\textbf{Flag} & \textbf{Meaning} \\ \hline
	\texttt{O\_RDONLY} & Open the file read-only \\ \hline
	\texttt{O\_WRONLY} & Open the file write-only \\ \hline
	\texttt{O\_RDWR} & Open the file for both reading and writing \\ \hline
	\texttt{O\_APPEND} & Append information to the end of the file \\ \hline
	\texttt{O\_TRUNC} & Initially clear all data from the file \\ \hline
	\texttt{O\_CREAT} & Create the file \\ \hline	
	\texttt{O\_EXCL} & If used with \texttt{O\_CREAT}, the caller MUST create the file; if the file exists it will fail \\ 	
\end{tabular}


\subsubsection*{Process Management}

\begin{lstlisting}[language=C]
pid_t fork( );
pid_t wait( int* status );
pid_t waitpid( pid_t pid, int *status, int options ); /* 0 for options fine */
void exit( int status );
\end{lstlisting}

\subsubsection*{Signals}

\begin{lstlisting}[language=C]
int kill( pid_t pid, int signal ); /* returns 0 returned if signal sent, -1 if an error */
int raise( int signal ); /* Send signal to the current process */
void ( *signal(int signum, void (*handler)(int)) ) (int); /* handle signal */ 
int pause( ) /* Suspend this program until a signal arrives */

int sigemptyset( sigset_t* set ); /* Initialize an empty sigset_t */
int sigaddset( sigset_t* set, int signal ); /* Add specified signal to set */
int sigfillset( sigset_t* set ); /* Add ALL signals to set */
int sigdelset( sigset_t* set, int signal ); /* Remove specified signal from set */ 
int sigismember( sigset_t* set, int signal ); /* Returns 1 if true, 0 if false */
int sigprocmask( int how, const sigset_t * set, sigset_t * old_set ); 
\end{lstlisting}

\subsubsection*{Message Queues}
\begin{lstlisting}[language=C]
key_t ftok( char *pathname, int proj );
int msgget( key_t key, int flag ); 
int msgsnd( int msqid, const void *ptr, size_t nbytes, int flag );
ssize_t msgrcv( int msqid, void *ptr, size_t nbytes, long type, int flag );
int msgctl( int msqid, int command, struct msqid_ds * buf ); /* IPC_RMID for command, NULL for buf */

/* IPC Structure can be any struct you like, as long as the first field is long */
struct ipc_msg {
  long mtype; /* Message type must be > 0 */
  char something[1]; /* Can be replaced with any type or structure */
};

\end{lstlisting}

\subsubsection*{Pipes and Shared Memory}

\begin{lstlisting}[language=C]
int pipe( int file_descriptors[] ); /* Array should be capacity 2 */
int shmget( key_t key, size_t size, int shmflg ); /* Can use IPC_PRIVATE for key, shmflag = UNIX permissions, can combine with IPC_CREAT and/or IPC_EXCL */
void* shmat( int shmid, const void* shmaddr, int shmflg ); /* NULL for shmaddr and 0 for shmflg for defaults */ 
int shmdt( const void* shmaddr );
int shmctl( int shmid, int cmd, struct shmid_ds *buf ); /* Use cmd = IPC_RMID and buf = NULL to delete */
void* mmap( void* address, size_t length, int protection, int flag, int fd, off_t offset ); /* address = NULL */
int mprotect( void* address, size_t length, int prot ); /* PROT_NONE, PROT_READ, PROT_WRITE, PROT_EXECUTE */
int msync( void* address, size_t length, int flags ); /* Use MS_SYNC for flags */
int munmap( void* address, size_t length );
\end{lstlisting}


\subsubsection*{Network}

\begin{lstlisting}[language=C]
int socket( int domain, int type, int protocol );
uint32_t htonl( uint32_t hostint32 ); /* Translate 4 byte int to network format */ 
uint16_t htons( uint16_t hostint16 ); /* Translate 2 byte int to network format */ 
uint32_t ntohl( uint32_t netint32 ); /* Translate 4 byte int to host format */ 
uint16_t ntohs( uint16_t netint16 ); /* Translate 2 byte int to host format */
int getaddrinfo( const char *node, const char *service, const struct addrinfo *hints, 
    struct addrinfo **res ); /* node = URL or IP, service = port */
int connect( int sockfd, struct sockaddr *addr, socklen_t len );
int bind( int sockfd, const struct sockaddr *addr, socklen_t addrlen );
int listen( int sockfd, int backlog );
int accept( int sockfd, struct sockaddr *addr, socklen_t *len );
int send( int sockfd, const void* msg, int length, int flags );
int recv( int sockfd, void * buffer, int length, int flags );
int sendto( int sfd, const void* msg, int len, unsigned int flags, const struct sockaddr* to, socklen_t token );
int recvfrom( int sfd, void* buffer, int len, unsigned int flags, struct sockaddr* from, int* fromlength );

CURLcode curl_global_init( long flags ); /* use CURL_GLOBAL_DEFAULT as flags */
CURL* curl_easy_init( );
void curl_easy_cleanup( CURL* handle );
CURLcode curl_easy_setopt( CURL *handle, CURLoption option, parameter ); /* See table below */
CURLcode curl_easy_perform( CURL * easy_handle );
CURLcode curl_easy_getinfo( CURL *curl, CURLINFO info, ... );
void curl_global_cleanup( );
CURLM *curl_multi_init( );
CURLMcode curl_multi_add_handle( CURLM *multi_handle, CURL *easy_handle );
CURLMcode curl_multi_remove_handle( CURLM *multi_handle, CURL *easy_handle );
CURLMcode curl_multi_perform( CURLM *multi_handle, int *running_handles );
CURLMcode curl_multi_wait(CURLM *multi_handle, struct curl_waitfd extra_fds[],
  unsigned int extra_nfds, int timeout_ms, int *numfds );
CURLMsg *curl_multi_info_read( CURLM *multi_handle, int *msgs_in_queue );
CURLMcode curl_multi_fdset( CURLM *mh, fd_set *rd_fd_set, fd_set *wr_fd_set, fd_set *ex_fd_set, int *max_fd );
CURLMcode curl_multi_timeout( CURLM *multi_handle, long *timeout );
size_t write_callback( char *ptr, size_t size, size_t nmemb, void *userdata );
size_t read_callback( char *buffer, size_t size, size_t nitems, void *inputdata );
\end{lstlisting}

Options for \texttt{curl\_easy\_setopt}:\\
\begin{tabular}{l|l|l}
	\textbf{CURLoption} & \textbf{Meaning} & \textbf{Parameter} \\ \hline
	\texttt{CURLOPT\_URL} & The URL to connect to & Character array \\ \hline
	\texttt{CURLOPT\_WRITEFUNCTION} & Set the write callback function & Function Pointer \\ \hline
	\texttt{CURLOPT\_WRITEDATA} & Set the write callback data & \texttt{void*} \\ \hline
	\texttt{CURLOPT\_READFUNCTION} & Set the read callback function & Function Pointer \\ \hline
	\texttt{CURLOPT\_READDATA} & Set the read callback data & \texttt{void*} \\ \hline
	\texttt{CURLOPT\_PUT} & Issue HTTP PUT request & \texttt{1L}\\ \hline
	\texttt{CURLOPT\_POST} & Issue HTTP POST request & \texttt{1L}\\ \hline
	\texttt{CURLOPT\_HTTPGET} & Issue HTTP GET request (this is the default, though) & \texttt{1L}\\ 
\end{tabular}

\newpage
\subsubsection*{Threads and Concurrency}

\begin{lstlisting}[language=C]
pthread_create( pthread_t *thread, const pthread_attr_t *attr, void *(*start_routine)( void * ), void *arg );
pthread_join( pthread_t thread, void **returnValue );
pthread_detach( pthread_t thread );
pthread_cancel( pthread_t thread );
pthread_testcancel( ); /* If the thread is cancelled, this function does not return (thread terminated) */
pthread_setcanceltype( int type, int *oldtype );
pthread_cleanup_push( void (*routine)(void*), void *argument ); /* Register cleanup handler, with argument */ 
pthread_cleanup_pop( int execute ); /* Run if execute is non-zero */
pthread_exit( void *value );

pthread_mutex_init( pthread_mutex_t *mutex, pthread_mutexattr_t *attributes );
pthread_mutex_lock( pthread_mutex_t *mutex );
pthread_mutex_trylock( pthread_mutex_t *mutex ); /* Returns 0 on success */
pthread_mutex_unlock( pthread_mutex_t *mutex );
pthread_mutex_destroy( pthread_mutex_t *mutex );
pthread_rwlock_init( pthread_rwlock_t * rwlock, pthread_rwlockattr_t * attr );
pthread_rwlock_rdlock( pthread_rwlock_t * rwlock );
pthread_rwlock_tryrdlock( pthread_rwlock_t * rwlock );
pthread_rwlock_wrlock( pthread_rwlock_t * rwlock );
pthread_rwlock_trywrlock( pthread_rwlock_t * rwlock );
pthread_rwlock_unlock( pthread_rwlock_t * rwlock );
pthread_rwlock_destroy( pthread_rwlock_t * rwlock );

sem_init( sem_t* semaphore, int shared, int initial_value); /* 0 for shared OK */
sem_destroy( sem_t* semaphore );
sem_wait( sem_t* semaphore );
sem_trywait( semt_t* semaphore );
sem_post( sem_t* semaphore );

pthread_cond_init( pthread_cond_t *cv, pthread_condattr_t *attributes );
pthread_cond_wait( pthread_cond_t *cv, pthread_mutex_t *mutex );
pthread_cond_signal( pthread_cond_t *cv );
pthread_cond_broadcast( pthread_cond_t *cv );
pthread_cond_destroy( pthread_cond_t *cv );
\end{lstlisting}

\subsubsection*{Atomic Types}
Atomic operations are defined on integral or pointer types of length 1, 2, 4, or 8; replace \texttt{type} with that type.

\begin{lstlisting}[language=C]
type __sync_lock_test_and_set( type *ptr, type value );
bool __sync_bool_compare_and_swap( type *ptr, type oldval, type newval );
type __sync_val_compare_and_swap( type *ptr, type oldval, type newval );
\end{lstlisting}


The following functions perform the operation and return the \textit{old} value:
\begin{lstlisting}[language=C]
type __sync_fetch_and_add( type *ptr, type value );
type __sync_fetch_and_sub( type *ptr, type value );
type __sync_fetch_and_or( type *ptr, type value );
type __sync_fetch_and_and( type *ptr, type value );
type __sync_fetch_and_xor( type *ptr, type value );
type __sync_fetch_and_nand( type *ptr, type value );
\end{lstlisting}

The following functions perform the operation and return the \textit{new} value:
\begin{lstlisting}[language=C]
type __sync_add_and_fetch( type *ptr, type value );
type __sync_sub_and_fetch( type *ptr, type value );
type __sync_or_and_fetch( type *ptr, type value );
type __sync_and_and_fetch( type *ptr, type value );
type __sync_xor_and_fetch( type *ptr, type value );
type __sync_nand_and_fetch( type *ptr, type value );
\end{lstlisting}


\subsubsection*{Select, Poll}

\begin{lstlisting}[language=C]
int select( int nfds, fd_set *readfds, fd_set *writefds, fd_set *exceptfds, struct timeval *timeout );
void FD_ZERO( fd_set *set ); /* Clear the set */
void FD_SET( int fd, fd_set *set ); /* Add fd to the set */
void FD_CLR( int fd, fd_set *set ); /* Remove fd from the set */
int FD_ISSET( int fd, fd_set *set ); /* Tests if fd is a part of the set */

struct timeval {
  long tv_sec; /* seconds */
  long tv_usec; /* microseconds */ 
};

int pselect( int nfds, fd_set *rd, fd_set *wr, fd_set *ex, const struct timespec *to, const sigset_t *mask );

struct timespec {
  long tv_sec; /* seconds */
  long tv_nsec; /* nanoseconds */
};

int poll( struct pollfd *fds, nfds_t nfds, int timeout );

struct pollfd {
  int fd; /* file descriptor */ 
  short events; /* requested events */ 
  short revents; /* returned events */
};
\end{lstlisting}

\newpage
\subsubsection*{AIO, Libevent}
\begin{lstlisting}[language=C]
struct aiocb {
  int aio_fildes;               /* File descriptor */
  off_t aio_offset;             /* Offset for I/O */
  volatile void* aio_buf;       /* Buffer */
  size_t aio_nbytes;            /* Number of bytes to transfer */
  int aio_reqprio;              /* Request priority */
  struct sigevent aio_sigevent; /* Signal Info */
  int aio_lio_opcode;           /* Operation for List I/O */
};

struct sigevent {
  int sigev_notify;                              /* Notify Type */
  int sigev_signo;                               /* Signal number */
  union sigval sigev_value;                      /* Notify argument */
  void (*sigev_notify_function) (union sigval);  /* Notify Function */
  pthread_attr_t *sigev_notify_attributes;       /* Notify attributes */
};

union sigval {
  int sival_int;
  void* sival_ptr;
};

int aio_read( struct aiocb* aiocb );
int aio_write( struct aiocb* aiocb );
int aio_error( const struct aiocb* aiocb );
ssize_t aio_return( const struct aiocb* aiocb );
int aio_suspend( const struct aiocb *const list[], int nent, const struct timespec* timeout );
int aio_cancel( int fd, struct aiocb* aiocb );
int lio_listio( int mode, struct aiocb * const list[ ], int nent, struct sigevent* sigev );

int evthread_use_pthreads( );
struct event_base* event_base_new( ); /* Create an event_base with default settings */
struct event_base* event_base_new_with_config( const struct event_config* cfg ); /* Create with configuration */
struct event_config* event_config_new( );
void event_config_free( struct event_config* cfg );
void event_base_free( struct event_base* base );
typedef void (*event_callback_fn)( evutil_socket_t fd, short what, void* arg );
struct event* event_new( struct event_base* b, evutil_socket_t fd, short what, event_callback_fn cb, void* arg );
void event_free( struct event* event );
int event_add( struct event* ev, const struct timeval* tv );
int event_del( struct event* ev );
int event_base_dispatch( struct event_base* base );
int event_base_loop( struct event_base *base, int flags );
int event_base_loopexit( struct event_base* base, const struct timeval* tv );
int event_base_loopbreak( struct event_base* base );
void libevent_global_shutdown( );
struct bufferevent* bufferevent_socket_new( struct event_base* base, evutil_socket_t fd, 
    enum bufferevent_options options );
void bufferevent_free( struct bufferevent* bev );

typedef void ( *bufferevent_data_cb )( struct bufferevent* bev, void* ctx );
typedef void ( *bufferevent_event_cb )( struct bufferevent* bev, short events, void* ctx );
void bufferevent_setcb( struct bufferevent* bufev, bufferevent_data_cb readcb, bufferevent_data_cb writecb, bufferevent_event_cb eventcb, void* cbarg );

int bufferevent_socket_connect( struct bufferevent* bev, struct sockaddr* address, int addrlen );

\end{lstlisting}

\end{document}